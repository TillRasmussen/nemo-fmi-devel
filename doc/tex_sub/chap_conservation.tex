\documentclass[../tex_main/NEMO_manual]{subfiles}
\begin{document}

% ================================================================
% Invariant of the Equations
% ================================================================
\chapter{Invariants of the Primitive Equations}
\label{chap:Invariant}
\minitoc

The continuous equations of motion have many analytic properties. Many 
quantities (total mass, energy, enstrophy, etc.) are strictly conserved in 
the inviscid and unforced limit, while ocean physics conserve the total 
quantities on which they act (momentum, temperature, salinity) but dissipate 
their total variance (energy, enstrophy, etc.). Unfortunately, the finite 
difference form of these equations is not guaranteed to retain all these 
important properties. In constructing the finite differencing schemes, we 
wish to ensure that certain integral constraints will be maintained. In 
particular, it is desirable to construct the finite difference equations so 
that horizontal kinetic energy and/or potential enstrophy of horizontally 
non-divergent flow, and variance of temperature and salinity will be 
conserved in the absence of dissipative effects and forcing. \citet{Arakawa1966} 
has first pointed out the advantage of this approach. He showed that if 
integral constraints on energy are maintained, the computation will be free 
of the troublesome "non linear" instability originally pointed out by 
\citet{Phillips1959}. A consistent formulation of the energetic properties is 
also extremely important in carrying out long-term numerical simulations for 
an oceanographic model. Such a formulation avoids systematic errors that 
accumulate with time \citep{Bryan1997}.

The general philosophy of OPA which has led to the discrete formulation 
presented in {\S}II.2 and II.3 is to choose second order non-diffusive 
scheme for advective terms for both dynamical and tracer equations. At this 
level of complexity, the resulting schemes are dispersive schemes. 
Therefore, they require the addition of a diffusive operator to be stable. 
The alternative is to use diffusive schemes such as upstream or flux 
corrected schemes. This last option was rejected because we prefer a 
complete handling of the model diffusion, i.e. of the model physics rather 
than letting the advective scheme produces its own implicit diffusion 
without controlling the space and time structure of this implicit diffusion. 
Note that in some very specific cases as passive tracer studies, the 
positivity of the advective scheme is required. In that case, and in that 
case only, the advective scheme used for passive tracer is a flux correction 
scheme \citep{Marti1992, Levy1996, Levy1998}.

% -------------------------------------------------------------------------------------------------------------
%       Conservation Properties on Ocean Dynamics
% -------------------------------------------------------------------------------------------------------------
\section{Conservation properties on ocean dynamics}
\label{sec:Invariant_dyn}

The non linear term of the momentum equations has been split into a 
vorticity term, a gradient of horizontal kinetic energy and a vertical 
advection term. Three schemes are available for the former (see {\S}~II.2) 
according to the CPP variable defined (default option\textbf{ 
}or \textbf{key{\_}vorenergy } or \textbf{key{\_}vorcombined 
} defined). They differ in their conservative 
properties (energy or enstrophy conserving scheme). The two latter terms 
preserve the total kinetic energy: the large scale kinetic energy is also 
preserved in practice. The remaining non-diffusive terms of the momentum 
equation (namely the hydrostatic and surface pressure gradient terms) also 
preserve the total kinetic energy and have no effect on the vorticity of the 
flow.

\textbf{* relative, planetary and total vorticity term:}

Let us define as either the relative, planetary and total potential 
vorticity, i.e. , , and , respectively. The continuous formulation of the 
vorticity term satisfies following integral constraints:
\begin{equation} \label{eq:vor_vorticity}
\int_D {{\textbf {k}}\cdot \frac{1}{e_3 }\nabla \times \left( {\varsigma 
\;{\rm {\bf k}}\times {\textbf {U}}_h } \right)\;dv} =0
\end{equation}

\begin{equation} \label{eq:vor_enstrophy}
if\quad \chi =0\quad \quad \int\limits_D {\varsigma \;{\textbf{k}}\cdot 
\frac{1}{e_3 }\nabla \times \left( {\varsigma {\textbf{k}}\times {\textbf{U}}_h } \right)\;dv} =-\int\limits_D {\frac{1}{2}\varsigma ^2\,\chi \;dv} 
=0
\end{equation}

\begin{equation} \label{eq:vor_energy}
\int_D {{\textbf{U}}_h \times \left( {\varsigma \;{\textbf{k}}\times {\textbf{U}}_h } \right)\;dv} =0
\end{equation}
where $dv = e_1\, e_2\, e_3\, di\, dj\, dk$ is the volume element. 
(II.4.1a) means that $\varsigma $ is conserved. (II.4.1b) is obtained by an 
integration by part. It means that $\varsigma^2$ is conserved for a horizontally 
non-divergent flow. 
(II.4.1c) is even satisfied locally since the vorticity term is orthogonal 
to the horizontal velocity. It means that the vorticity term has no 
contribution to the evolution of the total kinetic energy. (II.4.1a) is 
obviously always satisfied, but (II.4.1b) and (II.4.1c) cannot be satisfied 
simultaneously with a second order scheme. Using the symmetry or 
anti-symmetry properties of the operators (Eqs II.1.10 and 11), it can be 
shown that the scheme (II.2.11) satisfies (II.4.1b) but not (II.4.1c), while 
scheme (II.2.12) satisfies (II.4.1c) but not (II.4.1b) (see appendix C). 
Note that the enstrophy conserving scheme on total vorticity has been chosen 
as the standard discrete form of the vorticity term.

\textbf{* Gradient of kinetic energy / vertical advection}

In continuous formulation, the gradient of horizontal kinetic energy has no 
contribution to the evolution of the vorticity as the curl of a gradient is 
zero. This property is satisfied locally with the discrete form of both the 
gradient and the curl operator we have made (property (II.1.9)~). Another 
continuous property is that the change of horizontal kinetic energy due to 
vertical advection is exactly balanced by the change of horizontal kinetic 
energy due to the horizontal gradient of horizontal kinetic energy:

\begin{equation} \label{eq:keg_zad}
\int_D {{\textbf{U}}_h \cdot \nabla _h \left( {1/2\;{\textbf{U}}_h ^2} \right)\;dv} =-\int_D {{\textbf{U}}_h \cdot \frac{w}{e_3 }\;\frac{\partial 
{\textbf{U}}_h }{\partial k}\;dv}
\end{equation}

Using the discrete form given in {\S}II.2-a and the symmetry or 
anti-symmetry properties of the mean and difference operators, \autoref{eq:keg_zad} is 
demonstrated in the Appendix C. The main point here is that satisfying 
\autoref{eq:keg_zad} links the choice of the discrete forms of the vertical advection 
and of the horizontal gradient of horizontal kinetic energy. Choosing one 
imposes the other. The discrete form of the vertical advection given in 
{\S}II.2-a is a direct consequence of formulating the horizontal kinetic 
energy as $1/2 \left( \overline{u^2}^i + \overline{v^2}^j \right) $ in the gradient term.

\textbf{* hydrostatic pressure gradient term}

In continuous formulation, a pressure gradient has no contribution to the 
evolution of the vorticity as the curl of a gradient is zero. This 
properties is satisfied locally with the choice of discretization we have 
made (property (II.1.9)~). In addition, when the equation of state is linear 
(i.e. when an advective-diffusive equation for density can be derived from 
those of temperature and salinity) the change of horizontal kinetic energy 
due to the work of pressure forces is balanced by the change of potential 
energy due to buoyancy forces:

\begin{equation} \label{eq:hpg_pe}
\int_D {-\frac{1}{\rho _o }\left. {\nabla p^h} \right|_z \cdot {\textbf {U}}_h \;dv} \;=\;\int_D {\nabla .\left( {\rho \,{\textbf{U}}} \right)\;g\;z\;\;dv}
\end{equation}

Using the discrete form given in {\S}~II.2-a and the symmetry or 
anti-symmetry properties of the mean and difference operators, (II.4.3) is 
demonstrated in the Appendix C. The main point here is that satisfying 
(II.4.3) strongly constraints the discrete expression of the depth of 
$T$-points and of the term added to the pressure gradient in $s-$coordinates: the 
depth of a $T$-point, $z_T$, is defined as the sum the vertical scale 
factors at $w$-points starting from the surface.

\textbf{* surface pressure gradient term}

In continuous formulation, the surface pressure gradient has no contribution 
to the evolution of vorticity. This properties is trivially satisfied 
locally as (II.2.3) (the equation verified by $\psi$ has been 
derived from the discrete formulation of the momentum equations, vertical 
sum and curl. Nevertheless, the $\psi$-equation is solved numerically by an 
iterative solver (see {\S}~III.5), thus the property is only satisfied with 
the accuracy required on the solver. In addition, with the rigid-lid 
approximation, the change of horizontal kinetic energy due to the work of 
surface pressure forces is exactly zero:
\begin{equation} \label{eq:spg}
\int_D {-\frac{1}{\rho _o }\nabla _h } \left( {p_s } \right)\cdot {\textbf{U}}_h \;dv=0
\end{equation}

(II.4.4) is satisfied in discrete form only if the discrete barotropic 
streamfunction time evolution equation is given by (II.2.3) (see appendix 
C). This shows that (II.2.3) is the only way to compute the streamfunction, 
otherwise there is no guarantee that the surface pressure force work 
vanishes.

% -------------------------------------------------------------------------------------------------------------
%       Conservation Properties on Ocean Thermodynamics
% -------------------------------------------------------------------------------------------------------------
\section{Conservation properties on ocean thermodynamics}
\label{sec:Invariant_tra}

In continuous formulation, the advective terms of the tracer equations 
conserve the tracer content and the quadratic form of the tracer, i.e.
\begin{equation} \label{eq:tra_tra2}
\int_D {\nabla .\left( {T\;{\textbf{U}}} \right)\;dv} =0
\;\text{and}
\int_D {T\;\nabla .\left( {T\;{\textbf{U}}} \right)\;dv} =0
\end{equation}

The numerical scheme used ({\S}II.2-b) (equations in flux form, second order 
centred finite differences) satisfies (II.4.5) (see appendix C). Note that 
in both continuous and discrete formulations, there is generally no strict 
conservation of mass, since the equation of state is non linear with respect 
to $T$ and $S$. In practice, the mass is conserved with a very good accuracy. 

% -------------------------------------------------------------------------------------------------------------
%       Conservation Properties on Momentum Physics
% -------------------------------------------------------------------------------------------------------------
\subsection{Conservation properties on momentum physics}
\label{subsec:Invariant_dyn_physics}

\textbf{* lateral momentum diffusion term}

The continuous formulation of the horizontal diffusion of momentum satisfies 
the following integral constraints~:
\begin{equation} \label{eq:dynldf_dyn}
\int\limits_D {\frac{1}{e_3 }{\rm {\bf k}}\cdot \nabla \times \left[ {\nabla 
_h \left( {A^{lm}\;\chi } \right)-\nabla _h \times \left( {A^{lm}\;\zeta 
\;{\rm {\bf k}}} \right)} \right]\;dv} =0
\end{equation}

\begin{equation} \label{eq:dynldf_div}
\int\limits_D {\nabla _h \cdot \left[ {\nabla _h \left( {A^{lm}\;\chi } 
\right)-\nabla _h \times \left( {A^{lm}\;\zeta \;{\rm {\bf k}}} \right)} 
\right]\;dv} =0
\end{equation}

\begin{equation} \label{eq:dynldf_curl}
\int_D {{\rm {\bf U}}_h \cdot \left[ {\nabla _h \left( {A^{lm}\;\chi } 
\right)-\nabla _h \times \left( {A^{lm}\;\zeta \;{\rm {\bf k}}} \right)} 
\right]\;dv} \leqslant 0
\end{equation}

\begin{equation} \label{eq:dynldf_curl2}
\mbox{if}\quad A^{lm}=cste\quad \quad \int_D {\zeta \;{\rm {\bf k}}\cdot 
\nabla \times \left[ {\nabla _h \left( {A^{lm}\;\chi } \right)-\nabla _h 
\times \left( {A^{lm}\;\zeta \;{\rm {\bf k}}} \right)} \right]\;dv} 
\leqslant 0
\end{equation}

\begin{equation} \label{eq:dynldf_div2}
\mbox{if}\quad A^{lm}=cste\quad \quad \int_D {\chi \;\nabla _h \cdot \left[ 
{\nabla _h \left( {A^{lm}\;\chi } \right)-\nabla _h \times \left( 
{A^{lm}\;\zeta \;{\rm {\bf k}}} \right)} \right]\;dv} \leqslant 0
\end{equation}


(II.4.6a) and (II.4.6b) means that the horizontal diffusion of momentum 
conserve both the potential vorticity and the divergence of the flow, while 
Eqs (II.4.6c) to (II.4.6e) mean that it dissipates the energy, the enstrophy 
and the square of the divergence. The two latter properties are only 
satisfied when the eddy coefficients are horizontally uniform.

Using (II.1.8) and (II.1.9), and the symmetry or anti-symmetry properties of 
the mean and difference operators, it is shown that the discrete form of the 
lateral momentum diffusion given in {\S}II.2-c satisfies all the integral 
constraints (II.4.6) (see appendix C). In particular, when the eddy 
coefficients are horizontally uniform, a complete separation of vorticity 
and horizontal divergence fields is ensured, so that diffusion (dissipation) 
of vorticity (enstrophy) does not generate horizontal divergence (variance 
of the horizontal divergence) and \textit{vice versa}. When the vertical curl of the horizontal 
diffusion of momentum (discrete sense) is taken, the term associated to the 
horizontal gradient of the divergence is zero locally. When the horizontal 
divergence of the horizontal diffusion of momentum (discrete sense) is 
taken, the term associated to the vertical curl of the vorticity is zero 
locally. The resulting term conserves $\chi$ and dissipates 
$\chi^2$ when the 
eddy coefficient is horizontally uniform.

\textbf{* vertical momentum diffusion term}

As for the lateral momentum physics, the continuous form of the vertical 
diffusion of momentum satisfies following integral constraints~:

conservation of momentum, dissipation of horizontal kinetic energy

\begin{equation} \label{eq:dynzdf_dyn}
\begin{aligned}
& \int_D {\frac{1}{e_3 }}  \frac{\partial }{\partial k}\left( \frac{A^{vm}}{e_3 }\frac{\partial {\textbf{U}}_h }{\partial k} \right) \;dv = \overrightarrow{\textbf{0}} \\ 
& \int_D \textbf{U}_h \cdot \frac{1}{e_3} \frac{\partial}{\partial k} \left( {\frac{A^{vm}}{e_3 }}{\frac{\partial \textbf{U}_h }{\partial k}} \right) \;dv \leq 0 \\ 
 \end{aligned} 
 \end{equation}
conservation of vorticity, dissipation of enstrophy
\begin{equation} \label{eq:dynzdf_vor}
\begin{aligned}
& \int_D {\frac{1}{e_3 }{\rm {\bf k}}\cdot \nabla \times \left( {\frac{1}{e_3 
}\frac{\partial }{\partial k}\left( {\frac{A^{vm}}{e_3 }\frac{\partial {\rm 
{\bf U}}_h }{\partial k}} \right)} \right)\;dv} =0 \\ 
& \int_D {\zeta \,{\rm {\bf k}}\cdot \nabla \times \left( {\frac{1}{e_3 
}\frac{\partial }{\partial k}\left( {\frac{A^{vm}}{e_3 }\frac{\partial {\rm 
{\bf U}}_h }{\partial k}} \right)} \right)\;dv} \leq 0 \\ 
\end{aligned}
\end{equation}
conservation of horizontal divergence, dissipation of square of the 
horizontal divergence
\begin{equation} \label{eq:dynzdf_div}
\begin{aligned}
 &\int_D {\nabla \cdot \left( {\frac{1}{e_3 }\frac{\partial }{\partial 
k}\left( {\frac{A^{vm}}{e_3 }\frac{\partial {\rm {\bf U}}_h }{\partial k}} 
\right)} \right)\;dv} =0 \\ 
& \int_D {\chi \;\nabla \cdot \left( {\frac{1}{e_3 }\frac{\partial }{\partial 
k}\left( {\frac{A^{vm}}{e_3 }\frac{\partial {\rm {\bf U}}_h }{\partial k}} 
\right)} \right)\;dv} \leq 0 \\ 
\end{aligned}
\end{equation}

In discrete form, all these properties are satisfied in $z$-coordinate (see 
Appendix C). In $s$-coordinates, only first order properties can be 
demonstrated, i.e. the vertical momentum physics conserve momentum, 
potential vorticity, and horizontal divergence.

% -------------------------------------------------------------------------------------------------------------
%       Conservation Properties on Tracer Physics
% -------------------------------------------------------------------------------------------------------------
\subsection{Conservation properties on tracer physics}
\label{subsec:Invariant_tra_physics}

The numerical schemes used for tracer subgridscale physics are written in 
such a way that the heat and salt contents are conserved (equations in flux 
form, second order centred finite differences). As a form flux is used to 
compute the temperature and salinity, the quadratic form of these quantities 
(i.e. their variance) globally tends to diminish. As for the advective term, 
there is generally no strict conservation of mass even if, in practice, the 
mass is conserved with a very good accuracy. 

\textbf{* lateral physics: }conservation of tracer, dissipation of tracer 
variance, i.e.

\begin{equation} \label{eq:traldf_t_t2}
\begin{aligned}
&\int_D \nabla\, \cdot\, \left( A^{lT} \,\Re \,\nabla \,T \right)\;dv = 0 \\ 
&\int_D \,T\, \nabla\, \cdot\, \left( A^{lT} \,\Re \,\nabla \,T \right)\;dv \leq 0 \\ 
\end{aligned}
\end{equation}

\textbf{* vertical physics: }conservation of tracer, dissipation of tracer 
variance, i.e.

\begin{equation} \label{eq:trazdf_t_t2}
\begin{aligned}
& \int_D \frac{1}{e_3 } \frac{\partial }{\partial k}\left( \frac{A^{vT}}{e_3 }  \frac{\partial T}{\partial k}  \right)\;dv = 0 \\ 
& \int_D \,T \frac{1}{e_3 } \frac{\partial }{\partial k}\left( \frac{A^{vT}}{e_3 }  \frac{\partial T}{\partial k}  \right)\;dv \leq 0 \\ 
\end{aligned}
\end{equation}

Using the symmetry or anti-symmetry properties of the mean and difference 
operators, it is shown that the discrete form of tracer physics given in 
{\S}~II.2-c satisfies all the integral constraints (II.4.8) and (II.4.9) 
except the dissipation of the square of the tracer when non-geopotential 
diffusion is used (see appendix C). A discrete form of the lateral tracer 
physics can be derived which satisfies these last properties. Nevertheless, 
it requires a horizontal averaging of the vertical component of the lateral 
physics that prevents the use of implicit resolution in the vertical. It has 
not been implemented.

\end{document}
