\documentclass[../tex_main/NEMO_manual]{subfiles}
\begin{document}

% ================================================================
% Chapter � Appendix A : Curvilinear s-Coordinate Equations
% ================================================================
\chapter{Curvilinear $s-$Coordinate Equations}
\label{apdx:A}
\minitoc

\newpage
$\ $\newline    % force a new ligne

% ================================================================
% Chain rule
% ================================================================
\section{Chain rule for $s-$coordinates}
\label{sec:A_chain}

In order to establish the set of Primitive Equation in curvilinear $s$-coordinates
($i.e.$ an orthogonal curvilinear coordinate in the horizontal and an Arbitrary Lagrangian 
Eulerian (ALE) coordinate in the vertical), we start from the set of equations established 
in \autoref{subsec:PE_zco_Eq} for the special case $k = z$ and thus $e_3 = 1$, and we introduce 
an arbitrary vertical coordinate $a = a(i,j,z,t)$. Let us define a new vertical scale factor by 
$e_3 = \partial z / \partial s$ (which now depends on $(i,j,z,t)$) and the horizontal 
slope of $s-$surfaces by :
\begin{equation} \label{apdx:A_s_slope}
\sigma _1 =\frac{1}{e_1 }\;\left. {\frac{\partial z}{\partial i}} \right|_s 
\quad \text{and} \quad 
\sigma _2 =\frac{1}{e_2 }\;\left. {\frac{\partial z}{\partial j}} \right|_s 
\end{equation}

The chain rule to establish the model equations in the curvilinear $s-$coordinate 
system is:
\begin{equation} \label{apdx:A_s_chain_rule}
\begin{aligned}
&\left. {\frac{\partial \bullet }{\partial t}} \right|_z  =
\left. {\frac{\partial \bullet }{\partial t}} \right|_s 
	 -\frac{\partial \bullet }{\partial s}\;\frac{\partial s}{\partial t} \\
&\left. {\frac{\partial \bullet }{\partial i}} \right|_z  =
  \left. {\frac{\partial \bullet }{\partial i}} \right|_s 
  	  -\frac{\partial \bullet }{\partial s}\;\frac{\partial s}{\partial i}=
	  \left. {\frac{\partial \bullet }{\partial i}} \right|_s 
	  -\frac{e_1 }{e_3 }\sigma _1 \frac{\partial \bullet }{\partial s} \\
&\left. {\frac{\partial \bullet }{\partial j}} \right|_z  =
\left. {\frac{\partial \bullet }{\partial j}} \right|_s 
	- \frac{\partial \bullet }{\partial s}\;\frac{\partial s}{\partial j}=
\left. {\frac{\partial \bullet }{\partial j}} \right|_s 
	- \frac{e_2 }{e_3 }\sigma _2 \frac{\partial \bullet }{\partial s} \\
&\;\frac{\partial \bullet }{\partial z}  \;\; = \frac{1}{e_3 }\frac{\partial \bullet }{\partial s} \\
\end{aligned}
\end{equation}

In particular applying the time derivative chain rule to $z$ provides the expression 
for $w_s$,  the vertical velocity of the $s-$surfaces referenced to a fix z-coordinate:
\begin{equation} \label{apdx:A_w_in_s}
w_s   =  \left.   \frac{\partial z }{\partial t}   \right|_s 
				= \frac{\partial z}{\partial s} \; \frac{\partial s}{\partial t} 
			    = e_3 \, \frac{\partial s}{\partial t} 
\end{equation}


% ================================================================
% continuity equation
% ================================================================
\section{Continuity equation in $s-$coordinates}
\label{sec:A_continuity}

Using (\autoref{apdx:A_s_chain_rule}) and the fact that the horizontal scale factors 
$e_1$ and $e_2$ do not depend on the vertical coordinate, the divergence of 
the velocity relative to the ($i$,$j$,$z$) coordinate system is transformed as follows
in order to obtain its expression in the curvilinear $s-$coordinate system:

\begin{subequations} 
\begin{align*} {\begin{array}{*{20}l} 
\nabla \cdot {\rm {\bf U}} 
&= \frac{1}{e_1 \,e_2 }  \left[ \left. {\frac{\partial (e_2 \,u)}{\partial i}} \right|_z 
						+\left. {\frac{\partial(e_1 \,v)}{\partial j}} \right|_z  \right]
+ \frac{\partial w}{\partial z} 		\\
\\
&     = \frac{1}{e_1 \,e_2 }  \left[ 
		  \left.   \frac{\partial (e_2 \,u)}{\partial i}    \right|_s       
		  - \frac{e_1 }{e_3 } \sigma _1 \frac{\partial (e_2 \,u)}{\partial s}
		+ \left.   \frac{\partial (e_1 \,v)}{\partial j}    \right|_s       
		  - \frac{e_2 }{e_3 } \sigma _2 \frac{\partial (e_1 \,v)}{\partial s}	\right]
	+ \frac{\partial w}{\partial s} \; \frac{\partial s}{\partial z}                        \\
\\
&     = \frac{1}{e_1 \,e_2 }   \left[ 
		  \left.   \frac{\partial (e_2 \,u)}{\partial i}    \right|_s       
		+ \left.   \frac{\partial (e_1 \,v)}{\partial j}    \right|_s       	\right]
	+ \frac{1}{e_3 }\left[        \frac{\partial w}{\partial s}
						-  \sigma _1 \frac{\partial u}{\partial s}
						-  \sigma _2 \frac{\partial v}{\partial s}      \right]          \\
\\
&     = \frac{1}{e_1 \,e_2 \,e_3 }   \left[ 
		  \left.   \frac{\partial (e_2 \,e_3 \,u)}{\partial i}    \right|_s  
		  -\left.    e_2 \,u    \frac{\partial e_3 }{\partial i}     \right|_s      
		+ \left.  \frac{\partial (e_1 \,e_3 \,v)}{\partial j}    \right|_s
		  - \left.    e_1 v      \frac{\partial e_3 }{\partial j}    \right|_s   \right]          \\
& \qquad \qquad \qquad \qquad \qquad \qquad \qquad \qquad \qquad
	+ \frac{1}{e_3 } \left[        \frac{\partial w}{\partial s}
						-  \sigma _1 \frac{\partial u}{\partial s}
						-  \sigma _2 \frac{\partial v}{\partial s}      \right]      \\
%
\intertext{Noting that $
  \frac{1}{e_1} \left.{ \frac{\partial e_3}{\partial i}} \right|_s 
=\frac{1}{e_1} \left.{ \frac{\partial^2 z}{\partial i\,\partial s}} \right|_s 
=\frac{\partial}{\partial s} \left( {\frac{1}{e_1 } \left.{ \frac{\partial z}{\partial i} }\right|_s } \right)
=\frac{\partial \sigma _1}{\partial s}
$ and $
\frac{1}{e_2 }\left. {\frac{\partial e_3 }{\partial j}} \right|_s 
=\frac{\partial \sigma _2}{\partial s}
$, it becomes:}
%
\nabla \cdot {\rm {\bf U}} 
& = \frac{1}{e_1 \,e_2 \,e_3 }  \left[   
		  \left.  \frac{\partial (e_2 \,e_3 \,u)}{\partial i} \right|_s 
		+\left.  \frac{\partial (e_1 \,e_3 \,v)}{\partial j} \right|_s        \right]        \\ 
& \qquad \qquad \qquad \qquad \quad
 +\frac{1}{e_3 }\left[ {\frac{\partial w}{\partial s}-u\frac{\partial \sigma _1 }{\partial s}-v\frac{\partial \sigma _2 }{\partial s}-\sigma _1 \frac{\partial u}{\partial s}-\sigma _2 \frac{\partial v}{\partial s}} \right] \\ 
\\
& = \frac{1}{e_1 \,e_2 \,e_3 }  \left[   
		  \left.  \frac{\partial (e_2 \,e_3 \,u)}{\partial i} \right|_s 
		+\left.  \frac{\partial (e_1 \,e_3 \,v)}{\partial j} \right|_s        \right]      
   + \frac{1}{e_3 } \; \frac{\partial}{\partial s}   \left[  w -  u\;\sigma _1  - v\;\sigma _2  \right]
\end{array} } 		
\end{align*}
\end{subequations}

Here, $w$ is the vertical velocity relative to the $z-$coordinate system. 
Introducing the dia-surface velocity component, $\omega $, defined as 
the volume flux across the moving $s$-surfaces per unit horizontal area:
\begin{equation} \label{apdx:A_w_s}
\omega  = w - w_s - \sigma _1 \,u - \sigma _2 \,v    \\
\end{equation}
with $w_s$ given by \autoref{apdx:A_w_in_s}, we obtain the expression for 
the divergence of the velocity in the curvilinear $s-$coordinate system:
\begin{subequations} 
\begin{align*} {\begin{array}{*{20}l} 
\nabla \cdot {\rm {\bf U}} 
&= \frac{1}{e_1 \,e_2 \,e_3 }    \left[ 
		  \left.  \frac{\partial (e_2 \,e_3 \,u)}{\partial i} \right|_s 
		+\left.  \frac{\partial (e_1 \,e_3 \,v)}{\partial j} \right|_s        \right]      
+ \frac{1}{e_3 } \frac{\partial \omega }{\partial s} 
+ \frac{1}{e_3 } \frac{\partial w_s       }{\partial s}    \\
\\
&= \frac{1}{e_1 \,e_2 \,e_3 }    \left[ 
		  \left.  \frac{\partial (e_2 \,e_3 \,u)}{\partial i} \right|_s 
		+\left.  \frac{\partial (e_1 \,e_3 \,v)}{\partial j} \right|_s        \right]      
+ \frac{1}{e_3 } \frac{\partial \omega }{\partial s} 
+ \frac{1}{e_3 } \frac{\partial}{\partial s}  \left(  e_3 \; \frac{\partial s}{\partial t}   \right)   \\
\\
&= \frac{1}{e_1 \,e_2 \,e_3 }    \left[ 
		  \left.  \frac{\partial (e_2 \,e_3 \,u)}{\partial i} \right|_s 
		+\left.  \frac{\partial (e_1 \,e_3 \,v)}{\partial j} \right|_s        \right]      
+ \frac{1}{e_3 } \frac{\partial \omega }{\partial s} 
+ \frac{\partial}{\partial s} \frac{\partial s}{\partial t}
+ \frac{1}{e_3 } \frac{\partial s}{\partial t} \frac{\partial e_3}{\partial s}     \\
\\
&= \frac{1}{e_1 \,e_2 \,e_3 }    \left[ 
		  \left.  \frac{\partial (e_2 \,e_3 \,u)}{\partial i} \right|_s 
		+\left.  \frac{\partial (e_1 \,e_3 \,v)}{\partial j} \right|_s        \right]      
+ \frac{1}{e_3 } \frac{\partial \omega }{\partial s} 
+ \frac{1}{e_3 } \frac{\partial e_3}{\partial t}     \\
\end{array} } 		
\end{align*}
\end{subequations}

As a result, the continuity equation \autoref{eq:PE_continuity} in the 
$s-$coordinates is:
\begin{equation} \label{apdx:A_sco_Continuity}
\frac{1}{e_3 } \frac{\partial e_3}{\partial t} 
+ \frac{1}{e_1 \,e_2 \,e_3 }\left[ 
			{\left. {\frac{\partial (e_2 \,e_3 \,u)}{\partial i}} \right|_s 
		    +  \left. {\frac{\partial (e_1 \,e_3 \,v)}{\partial j}} \right|_s } \right]
 +\frac{1}{e_3 }\frac{\partial \omega }{\partial s} = 0   
\end{equation}
A additional term has appeared that take into account the contribution of the time variation 
of the vertical coordinate to the volume budget.


% ================================================================
% momentum equation
% ================================================================
\section{Momentum equation in $s-$coordinate}
\label{sec:A_momentum}

Here we only consider the first component of the momentum equation, 
the generalization to the second one being straightforward.

$\ $\newline    % force a new ligne

$\bullet$ \textbf{Total derivative in vector invariant form}

Let us consider \autoref{eq:PE_dyn_vect}, the first component of the momentum 
equation in the vector invariant form. Its total $z-$coordinate time derivative, 
$\left. \frac{D u}{D t} \right|_z$ can be transformed as follows in order to obtain 
its expression in the curvilinear $s-$coordinate system:

\begin{subequations} 
\begin{align*} {\begin{array}{*{20}l} 
\left. \frac{D u}{D t} \right|_z 
&= \left. {\frac{\partial u }{\partial t}} \right|_z 
   - \left. \zeta \right|_z v 
  + \frac{1}{2e_1} \left.{ \frac{\partial (u^2+v^2)}{\partial i}} \right|_z 
  + w \;\frac{\partial u}{\partial z} \\
\\
&= \left. {\frac{\partial u }{\partial t}} \right|_z 
   - \left. \zeta \right|_z v 
  +  \frac{1}{e_1 \,e_2 }\left[ { \left.{ \frac{\partial (e_2 \,v)}{\partial i} }\right|_z 
                                             -\left.{ \frac{\partial (e_1 \,u)}{\partial j} }\right|_z } \right] \; v     
  +  \frac{1}{2e_1} \left.{ \frac{\partial (u^2+v^2)}{\partial i} } \right|_z 
  +  w \;\frac{\partial u}{\partial z}      \\
%
\intertext{introducing the chain rule (\autoref{apdx:A_s_chain_rule}) }
%
&= \left. {\frac{\partial u }{\partial t}} \right|_z       
   - \frac{1}{e_1\,e_2}\left[ { \left.{ \frac{\partial (e_2 \,v)}{\partial i} } \right|_s 
                                          -\left.{ \frac{\partial (e_1 \,u)}{\partial j} } \right|_s } \right.
                                          \left. {-\frac{e_1}{e_3}\sigma _1 \frac{\partial (e_2 \,v)}{\partial s}
                                                   +\frac{e_2}{e_3}\sigma _2 \frac{\partial (e_1 \,u)}{\partial s}} \right] \; v  \\ 
& \qquad \qquad \qquad \qquad
 { + \frac{1}{2e_1} \left(                                  \left.  \frac{\partial (u^2+v^2)}{\partial i} \right|_s 
                                    - \frac{e_1}{e_3}\sigma _1 \frac{\partial (u^2+v^2)}{\partial s}               \right)
   + \frac{w}{e_3 } \;\frac{\partial u}{\partial s} }    \\
\\
&= \left. {\frac{\partial u }{\partial t}} \right|_z       
  + \left. \zeta \right|_s \;v
  + \frac{1}{2\,e_1}\left. {\frac{\partial (u^2+v^2)}{\partial i}} \right|_s      \\
&\qquad \qquad \qquad \quad
  + \frac{w}{e_3 } \;\frac{\partial u}{\partial s}
   - \left[   {\frac{\sigma _1 }{e_3 }\frac{\partial v}{\partial s}
               - \frac{\sigma_2 }{e_3 }\frac{\partial u}{\partial s}}   \right]\;v      
   - \frac{\sigma _1 }{2e_3 }\frac{\partial (u^2+v^2)}{\partial s}      \\
\\
&= \left. {\frac{\partial u }{\partial t}} \right|_z       
  + \left. \zeta \right|_s \;v
  + \frac{1}{2\,e_1}\left. {\frac{\partial (u^2+v^2)}{\partial i}} \right|_s      \\
&\qquad \qquad \qquad \quad
 + \frac{1}{e_3} \left[    {w\frac{\partial u}{\partial s}
	                        +\sigma _1 v\frac{\partial v}{\partial s} - \sigma _2 v\frac{\partial u}{\partial s}
	                        - \sigma _1 u\frac{\partial u}{\partial s} - \sigma _1 v\frac{\partial v}{\partial s}} \right] \\
\\
&= \left. {\frac{\partial u }{\partial t}} \right|_z       
  + \left. \zeta \right|_s \;v
  + \frac{1}{2\,e_1}\left. {\frac{\partial (u^2+v^2)}{\partial i}} \right|_s  
  + \frac{1}{e_3} \left[  w - \sigma _2 v - \sigma _1 u  \right] 
		  			 \; \frac{\partial u}{\partial s}   \\
%
\intertext{Introducing $\omega$, the dia-a-surface velocity given by (\autoref{apdx:A_w_s}) }
%
&= \left. {\frac{\partial u }{\partial t}} \right|_z       
  + \left. \zeta \right|_s \;v
  + \frac{1}{2\,e_1}\left. {\frac{\partial (u^2+v^2)}{\partial i}} \right|_s  
  + \frac{1}{e_3 } \left( \omega - w_s \right) \frac{\partial u}{\partial s}   \\
\end{array} } 		
\end{align*}
\end{subequations}
%
Applying the time derivative chain rule (first equation of (\autoref{apdx:A_s_chain_rule}))
to $u$ and using (\autoref{apdx:A_w_in_s}) provides the expression of the last term 
of the right hand side,
\begin{equation*} {\begin{array}{*{20}l} 
w_s  \;\frac{\partial u}{\partial s} 
	= \frac{\partial s}{\partial t} \;  \frac{\partial u }{\partial s}
	= \left. {\frac{\partial u }{\partial t}} \right|_s  - \left. {\frac{\partial u }{\partial t}} \right|_z \quad , 
\end{array} } 		
\end{equation*}
leads to the $s-$coordinate formulation of the total $z-$coordinate time derivative, 
$i.e.$ the total $s-$coordinate time derivative :
\begin{align} \label{apdx:A_sco_Dt_vect}
\left. \frac{D u}{D t} \right|_s 
  = \left. {\frac{\partial u }{\partial t}} \right|_s       
  + \left. \zeta \right|_s \;v
  + \frac{1}{2\,e_1}\left. {\frac{\partial (u^2+v^2)}{\partial i}} \right|_s  
  + \frac{1}{e_3 } \omega \;\frac{\partial u}{\partial s}   
\end{align}
Therefore, the vector invariant form of the total time derivative has exactly the same 
mathematical form in $z-$ and $s-$coordinates. This is not the case for the flux form
as shown in next paragraph.

$\ $\newline    % force a new ligne

$\bullet$ \textbf{Total derivative in flux form}

Let us start from the total time derivative in the curvilinear $s-$coordinate system 
we have just establish. Following the procedure used to establish (\autoref{eq:PE_flux_form}), 
it can be transformed into :
%\begin{subequations} 
\begin{align*} {\begin{array}{*{20}l} 
\left. \frac{D u}{D t} \right|_s  &= \left. {\frac{\partial u }{\partial t}} \right|_s  
           					    & -  \zeta \;v 
           					+ \frac{1}{2\;e_1 } \frac{\partial \left( {u^2+v^2} \right)}{\partial i}
                                                 + \frac{1}{e_3} \omega \;\frac{\partial u}{\partial s}          \\
\\
  &= \left. {\frac{\partial u }{\partial t}} \right|_s  
          &+\frac{1}{e_1\;e_2}  \left(    \frac{\partial \left( {e_2 \,u\,u } \right)}{\partial i}
		                                    + \frac{\partial \left( {e_1 \,u\,v } \right)}{\partial j}     \right)
            + \frac{1}{e_3 } \frac{\partial \left( {\omega\,u} \right)}{\partial s}                                \\ 
\\
        &&- \,u \left[     \frac{1}{e_1 e_2 } \left(    \frac{\partial(e_2 u)}{\partial i}
				  	  	                 + \frac{\partial(e_1 v)}{\partial j}    \right)
                          + \frac{1}{e_3}        \frac{\partial \omega}{\partial s}                       \right]      \\
\\
        &&- \frac{v}{e_1 e_2 }\left(    v	\;\frac{\partial e_2 }{\partial i}
				 	           -u	\;\frac{\partial e_1 }{\partial j} 	\right)                             \\
\end{array} } 		
\end{align*}
%
Introducing the vertical scale factor inside the horizontal derivative of the first two terms 
($i.e.$ the horizontal divergence), it becomes :
\begin{subequations} 
\begin{align*} {\begin{array}{*{20}l} 
%\begin{align*} {\begin{array}{*{20}l} 
%{\begin{array}{*{20}l} 
\left. \frac{D u}{D t} \right|_s  
   &= \left. {\frac{\partial u }{\partial t}} \right|_s  
   &+ \frac{1}{e_1\,e_2\,e_3}  \left(  \frac{\partial( e_2 e_3 \,u^2 )}{\partial i}
			                  		  + \frac{\partial( e_1 e_3 \,u v )}{\partial j}     
	     							   -  e_2 u u \frac{\partial e_3}{\partial i}
							  -  e_1 u v \frac{\partial e_3 }{\partial j}    \right)
	    + \frac{1}{e_3} \frac{\partial \left( {\omega\,u} \right)}{\partial s}                                  \\
\\
           && - \,u \left[  \frac{1}{e_1 e_2 e_3} \left(   \frac{\partial(e_2 e_3 \, u)}{\partial i} 
			   							 + \frac{\partial(e_1 e_3 \, v)}{\partial j}  
	                							 -  e_2 u \;\frac{\partial e_3 }{\partial i}
				 	  	                      -  e_1 v \;\frac{\partial e_3 }{\partial j}   \right)
             -\frac{1}{e_3}        \frac{\partial \omega}{\partial s}                       \right]                      \\
\\
            && - \frac{v}{e_1 e_2 }\left( 	v  \;\frac{\partial e_2 }{\partial i}
				     	              -u  \;\frac{\partial e_1 }{\partial j} 	\right)                      \\
\\
   &= \left. {\frac{\partial u }{\partial t}} \right|_s  
   &+ \frac{1}{e_1\,e_2\,e_3}  \left(  \frac{\partial( e_2 e_3 \,u\,u )}{\partial i}
			                  		  + \frac{\partial( e_1 e_3 \,u\,v )}{\partial j}    \right)
     + \frac{1}{e_3 } \frac{\partial \left( {\omega\,u} \right)}{\partial s}                               \\
\\
&& - \,u \left[  \frac{1}{e_1 e_2 e_3} \left(   \frac{\partial(e_2 e_3 \, u)}{\partial i} 
									+ \frac{\partial(e_1 e_3 \, v)}{\partial j}  \right)
        -\frac{1}{e_3}        \frac{\partial \omega}{\partial s}                       \right]                  
     - \frac{v}{e_1 e_2 }\left( 	v   \;\frac{\partial e_2 }{\partial i}
				            	      -u   \;\frac{\partial e_1 }{\partial j} 	\right)                  \\
%
\intertext {Introducing a more compact form for the divergence of the momentum fluxes, 
and using (\autoref{apdx:A_sco_Continuity}), the $s-$coordinate continuity equation, 
it becomes : }
%
   &= \left. {\frac{\partial u }{\partial t}} \right|_s  
   &+ \left.  \nabla \cdot \left(   {{\rm {\bf U}}\,u}   \right)    \right|_s
     + \,u \frac{1}{e_3 } \frac{\partial e_3}{\partial t}    
      - \frac{v}{e_1 e_2 }\left(    v  \;\frac{\partial e_2 }{\partial i}
				             -u  \;\frac{\partial e_1 }{\partial j} 	\right) \\
\end{array} } 		
\end{align*}
\end{subequations}
which leads to the $s-$coordinate flux formulation of the total $s-$coordinate time derivative, 
$i.e.$ the total $s-$coordinate time derivative in flux form :
\begin{flalign}\label{apdx:A_sco_Dt_flux}
\left. \frac{D u}{D t} \right|_s   = \frac{1}{e_3}  \left. \frac{\partial ( e_3\,u)}{\partial t} \right|_s  
           + \left.  \nabla \cdot \left(   {{\rm {\bf U}}\,u}   \right)    \right|_s
           - \frac{v}{e_1 e_2 }\left(    v  \;\frac{\partial e_2 }{\partial i}
				 	          -u  \;\frac{\partial e_1 }{\partial j}            \right)
\end{flalign}
which is the total time derivative expressed in the curvilinear $s-$coordinate system.
It has the same form as in the $z-$coordinate but for the vertical scale factor 
that has appeared inside the time derivative which comes from the modification 
of (\autoref{apdx:A_sco_Continuity}), the continuity equation.

$\ $\newline    % force a new ligne

$\bullet$ \textbf{horizontal pressure gradient}

The horizontal pressure gradient term can be transformed as follows:
\begin{equation*}
\begin{split}
 -\frac{1}{\rho _o \, e_1 }\left. {\frac{\partial p}{\partial i}} \right|_z
 & =-\frac{1}{\rho _o e_1 }\left[ {\left. {\frac{\partial p}{\partial i}} \right|_s -\frac{e_1 }{e_3 }\sigma _1 \frac{\partial p}{\partial s}} \right] \\
& =-\frac{1}{\rho _o \,e_1 }\left. {\frac{\partial p}{\partial i}} \right|_s +\frac{\sigma _1 }{\rho _o \,e_3 }\left( {-g\;\rho \;e_3 } \right) \\
&=-\frac{1}{\rho _o \,e_1 }\left. {\frac{\partial p}{\partial i}} \right|_s -\frac{g\;\rho }{\rho _o }\sigma _1
\end{split}
\end{equation*}
Applying similar manipulation to the second component and replacing 
$\sigma _1$ and $\sigma _2$ by their expression \autoref{apdx:A_s_slope}, it comes:
\begin{equation} \label{apdx:A_grad_p_1}
\begin{split}
 -\frac{1}{\rho _o \, e_1 } \left. {\frac{\partial p}{\partial i}} \right|_z
&=-\frac{1}{\rho _o \,e_1 } \left(     \left.              {\frac{\partial p}{\partial i}} \right|_s 
                                                  + g\;\rho  \;\left. {\frac{\partial z}{\partial i}} \right|_s    \right) \\
%
 -\frac{1}{\rho _o \, e_2 }\left. {\frac{\partial p}{\partial j}} \right|_z
&=-\frac{1}{\rho _o \,e_2 } \left(    \left.               {\frac{\partial p}{\partial j}} \right|_s 
                                                   + g\;\rho \;\left. {\frac{\partial z}{\partial j}} \right|_s   \right) \\
\end{split}
\end{equation}

An additional term appears in (\autoref{apdx:A_grad_p_1}) which accounts for the 
tilt of $s-$surfaces with respect to geopotential $z-$surfaces.

As in $z$-coordinate, the horizontal pressure gradient can be split in two parts
following \citet{Marsaleix_al_OM08}. Let defined a density anomaly, $d$, by $d=(\rho - \rho_o)/ \rho_o$,
and a hydrostatic pressure anomaly, $p_h'$, by $p_h'= g \; \int_z^\eta d \; e_3 \; dk$. 
The pressure is then given by:
\begin{equation*} 
\begin{split}
p &= g\; \int_z^\eta \rho \; e_3 \; dk = g\; \int_z^\eta \left(  \rho_o \, d + 1 \right) \; e_3 \; dk   \\
   &= g \, \rho_o \; \int_z^\eta d \; e_3 \; dk + g \, \int_z^\eta e_3 \; dk    
\end{split}
\end{equation*}
Therefore, $p$ and $p_h'$ are linked through:
\begin{equation} \label{apdx:A_pressure}
   p = \rho_o \; p_h' + g \, ( z + \eta )
\end{equation}
and the hydrostatic pressure balance expressed in terms of $p_h'$ and $d$ is:
\begin{equation*} 
\frac{\partial p_h'}{\partial k} = - d \, g \, e_3
\end{equation*}

Substituing \autoref{apdx:A_pressure} in \autoref{apdx:A_grad_p_1} and using the definition of 
the density anomaly it comes the expression in two parts:
\begin{equation} \label{apdx:A_grad_p_2}
\begin{split}
 -\frac{1}{\rho _o \, e_1 } \left. {\frac{\partial p}{\partial i}} \right|_z
&=-\frac{1}{e_1 } \left(     \left.              {\frac{\partial p_h'}{\partial i}} \right|_s 
                                       + g\; d  \;\left. {\frac{\partial z}{\partial i}} \right|_s    \right)  - \frac{g}{e_1 } \frac{\partial \eta}{\partial i} \\
%
 -\frac{1}{\rho _o \, e_2 }\left. {\frac{\partial p}{\partial j}} \right|_z
&=-\frac{1}{e_2 } \left(    \left.               {\frac{\partial p_h'}{\partial j}} \right|_s 
                                        + g\; d \;\left. {\frac{\partial z}{\partial j}} \right|_s   \right)  - \frac{g}{e_2 } \frac{\partial \eta}{\partial j}\\
\end{split}
\end{equation}
This formulation of the pressure gradient is characterised by the appearance of a term depending on the 
the sea surface height only (last term on the right hand side of expression \autoref{apdx:A_grad_p_2}).
This term will be loosely termed \textit{surface pressure gradient}
whereas the first term will be termed the 
\textit{hydrostatic pressure gradient} by analogy to the $z$-coordinate formulation. 
In fact, the the true surface pressure gradient is $1/\rho_o \nabla (\rho \eta)$, and 
$\eta$ is implicitly included in the computation of $p_h'$ through the upper bound of 
the vertical integration.
 

$\ $\newline    % force a new ligne

$\bullet$ \textbf{The other terms of the momentum equation}

The coriolis and forcing terms as well as the the vertical physics remain unchanged 
as they involve neither time nor space derivatives. The form of the lateral physics is 
discussed in \autoref{apdx:B}.


$\ $\newline    % force a new ligne

$\bullet$ \textbf{Full momentum equation}

To sum up, in a curvilinear $s$-coordinate system, the vector invariant momentum equation 
solved by the model has the same mathematical expression as the one in a curvilinear 
$z-$coordinate, except for the pressure gradient term :
\begin{subequations} \label{apdx:A_dyn_vect}
\begin{multline} \label{apdx:A_PE_dyn_vect_u}
 \frac{\partial u}{\partial t}=
	+   \left( {\zeta +f} \right)\,v                                    
	-   \frac{1}{2\,e_1} \frac{\partial}{\partial i} \left(  u^2+v^2   \right) 
	-   \frac{1}{e_3} \omega \frac{\partial u}{\partial k}       \\
        -   \frac{1}{e_1 } \left(    \frac{\partial p_h'}{\partial i} + g\; d  \; \frac{\partial z}{\partial i}    \right)  
        -   \frac{g}{e_1 } \frac{\partial \eta}{\partial i}
	+   D_u^{\vect{U}}  +   F_u^{\vect{U}}
\end{multline}
\begin{multline} \label{apdx:A_dyn_vect_v}
\frac{\partial v}{\partial t}=
	-   \left( {\zeta +f} \right)\,u   
	-   \frac{1}{2\,e_2 }\frac{\partial }{\partial j}\left(  u^2+v^2  \right)     	
	-   \frac{1}{e_3 } \omega \frac{\partial v}{\partial k}         \\
        -   \frac{1}{e_2 } \left(    \frac{\partial p_h'}{\partial j} + g\; d  \; \frac{\partial z}{\partial j}    \right)  
        -   \frac{g}{e_2 } \frac{\partial \eta}{\partial j}
	+  D_v^{\vect{U}}  +   F_v^{\vect{U}}
\end{multline}
\end{subequations}
whereas the flux form momentum equation differ from it by the formulation of both
the time derivative and the pressure gradient term  :
\begin{subequations} \label{apdx:A_dyn_flux}
\begin{multline} \label{apdx:A_PE_dyn_flux_u}
 \frac{1}{e_3} \frac{\partial \left(  e_3\,u  \right) }{\partial t} =
	\nabla \cdot \left(   {{\rm {\bf U}}\,u}   \right) 
	+   \left\{ {f + \frac{1}{e_1 e_2 }\left(    v  \;\frac{\partial e_2 }{\partial i}
				 	                        -u  \;\frac{\partial e_1 }{\partial j}            \right)} \right\} \,v     \\                               
        -   \frac{1}{e_1 } \left(    \frac{\partial p_h'}{\partial i} + g\; d  \; \frac{\partial z}{\partial i}    \right)  
        -   \frac{g}{e_1 } \frac{\partial \eta}{\partial i}
	+   D_u^{\vect{U}}  +   F_u^{\vect{U}}
\end{multline}
\begin{multline} \label{apdx:A_dyn_flux_v}
 \frac{1}{e_3}\frac{\partial \left(  e_3\,v  \right) }{\partial t}=
	-  \nabla \cdot \left(   {{\rm {\bf U}}\,v}   \right) 
	+   \left\{ {f + \frac{1}{e_1 e_2 }\left(    v  \;\frac{\partial e_2 }{\partial i}
				 	                         -u  \;\frac{\partial e_1 }{\partial j}            \right)} \right\} \,u     \\                               
        -   \frac{1}{e_2 } \left(    \frac{\partial p_h'}{\partial j} + g\; d  \; \frac{\partial z}{\partial j}    \right)  
        -   \frac{g}{e_2 } \frac{\partial \eta}{\partial j}
	+  D_v^{\vect{U}}  +   F_v^{\vect{U}}
\end{multline}
\end{subequations}
Both formulation share the same hydrostatic pressure balance expressed in terms of
hydrostatic pressure and density anomalies, $p_h'$ and $d=( \frac{\rho}{\rho_o}-1 )$:
\begin{equation} \label{apdx:A_dyn_zph}
\frac{\partial p_h'}{\partial k} = - d \, g \, e_3
\end{equation}

It is important to realize that the change in coordinate system has only concerned
the position on the vertical. It has not affected (\textbf{i},\textbf{j},\textbf{k}), the 
orthogonal curvilinear set of unit vectors. ($u$,$v$) are always horizontal velocities
so that their evolution is driven by \emph{horizontal} forces, in particular 
the pressure gradient. By contrast, $\omega$ is not $w$, the third component of the velocity,
but the dia-surface velocity component, $i.e.$ the volume flux across the moving 
$s$-surfaces per unit horizontal area. 


% ================================================================
% Tracer equation
% ================================================================
\section{Tracer equation}
\label{sec:A_tracer}

The tracer equation is obtained using the same calculation as for the continuity 
equation and then regrouping the time derivative terms in the left hand side :

\begin{multline} \label{apdx:A_tracer}
 \frac{1}{e_3} \frac{\partial \left(  e_3 T  \right)}{\partial t} 
 	= -\frac{1}{e_1 \,e_2 \,e_3} 
		\left[           \frac{\partial }{\partial i} \left( {e_2 \,e_3 \;Tu} \right) 
		             +   \frac{\partial }{\partial j} \left( {e_1 \,e_3 \;Tv} \right)               \right]       \\
 	+  \frac{1}{e_3}  \frac{\partial }{\partial k} \left(                   Tw  \right)  
	 +  D^{T} +F^{T}
\end{multline}


The expression for the advection term is a straight consequence of (A.4), the 
expression of the 3D divergence in the $s-$coordinates established above. 

\end{document}
