\documentclass[../tex_main/NEMO_manual]{subfiles}
\begin{document}
% ================================================================
% Appendix D � Coding Rules
% ================================================================
\chapter{Coding Rules}
\label{apdx:D}
\minitoc

\newpage
$\ $\newline    % force a new ligne
$\ $\newline    % force a new ligne


A "model life" is more than ten years. Its software, composed of a few 
hundred modules, is used by many people who are scientists or students 
and do not necessarily know every aspect of computing very well. 
Moreover, a well thought-out program is easier to read and understand, 
less difficult to modify, produces fewer bugs and is easier to maintain. 
Therefore, it is essential that the model development follows some rules :

- well planned and designed

- well written

- well documented (both on- and off-line)

- maintainable

- easily portable

- flexible.

To satisfy part of these aims, \NEMO is written with a coding standard which 
is close to the ECMWF rules, named DOCTOR \citep{Gibson_TR86}. 
These rules present some advantages like :

- to provide a well presented program

- to use rules for variable names which allow recognition of their type 
(integer, real, parameter, local or shared variables, etc. ). 

This facilitates both the understanding and the debugging of an algorithm.

% ================================================================
% The program structure
% ================================================================
\section{Program structure}
\label{sec:D_structure}

Each program begins with a set of headline comments containing :

- the program title

- the purpose of the routine

- the method and algorithms used

- the detail of input and output interfaces

- the external routines and functions used (if they exist)

- references (if they exist)

- the author name(s), the date of creation and any updates.

- Each program is split into several well separated sections and 
sub-sections with an underlined title and specific labelled statements.

- A program has not more than 200 to 300 lines.

A template of a module style can be found on the NEMO depository 
in the following file : NEMO/OPA\_SRC/module\_example.
% ================================================================
% Coding conventions
% ================================================================
\section{Coding conventions}
\label{sec:D_coding}

- Use of the universal language \textsc{Fortran} 90, and try to avoid obsolescent
features like statement functions, do not use GO TO and EQUIVALENCE statements.

- A continuation line begins with the character {\&} indented by three spaces 
compared to the previous line, while the previous line ended with the character {\&}.

- All the variables must be declared. The code is usually compiled with implicit none.
 
- Never use continuation lines in the declaration of a variable. When searching a 
variable in the code through a \textit{grep} command, the declaration line will be found.

- In the declaration of a PUBLIC variable, the comment part at the end of the line 
should start with the two characters "\verb?!:?". the following UNIX command, \\
\verb?grep var_name *90 \ grep \!: ? \\
will display the module name and the line where the var\_name declaration is.

- Always use a three spaces indentation in DO loop, CASE, or IF-ELSEIF-ELSE-ENDIF 
statements.

- use a space after a comma, except when it appears to separate the indices of an array.

- use call to ctl\_stop routine instead of just a STOP.


\newpage
% ================================================================
% Naming Conventions
% ================================================================
\section{Naming conventions}
\label{sec:D_naming}

The purpose of the naming conventions is to use prefix letters to classify 
model variables. These conventions allow the variable type to be easily 
known and rapidly identified. The naming conventions are summarised 
in the Table below:


%--------------------------------------------------TABLE--------------------------------------------------
\begin{table}[htbp]  \label{tab:VarName}
\begin{center}
\begin{tabular}{|p{45pt}|p{35pt}|p{45pt}|p{40pt}|p{40pt}|p{40pt}|p{40pt}|p{40pt}|}
\hline  Type \par / Status &   integer&   real&   logical &   character  & structure &   double \par precision&   complex \\  
\hline
public  \par or  \par module variable& 
\textbf{m n} \par \textit{but not} \par \textbf{nn\_ np\_}& 
\textbf{a b e f g h o q r} \par \textbf{t} \textit{to} \textbf{x} \par but not \par \textbf{fs rn\_}& 
\textbf{l} \par \textit{but not} \par \textbf{lp ld} \par \textbf{ ll ln\_}& 
\textbf{c} \par \textit{but not} \par \textbf{cp cd} \par \textbf{cl cn\_}& 
\textbf{s} \par \textit{but not} \par \textbf{sd sd} \par \textbf{sl sn\_}& 
\textbf{d} \par \textit{but not} \par \textbf{dp dd} \par \textbf{dl dn\_}& 
\textbf{y} \par \textit{but not} \par \textbf{yp yd} \par \textbf{yl yn} \\
\hline
dummy \par argument& 
\textbf{k} \par \textit{but not} \par \textbf{kf}& 
\textbf{p} \par \textit{but not} \par \textbf{pp pf}& 
\textbf{ld}& 
\textbf{cd}& 
\textbf{sd}& 
\textbf{dd}& 
\textbf{yd} \\
\hline
local \par variable& 
\textbf{i}& 
\textbf{z}& 
\textbf{ll}& 
\textbf{cl}& 
\textbf{sl}& 
\textbf{dl}& 
\textbf{yl} \\
\hline
loop \par control& 
\textbf{j} \par \textit{but not} \par \textbf{jp}& 
& 
& 
&
& 
& 
 \\
\hline
parameter& 
\textbf{jp np\_}& 
\textbf{pp}& 
\textbf{lp}& 
\textbf{cp}& 
\textbf{sp}& 
\textbf{dp}& 
\textbf{yp} \\
\hline

namelist&
\textbf{nn\_}& 
\textbf{rn\_}& 
\textbf{ln\_}& 
\textbf{cn\_}& 
\textbf{sn\_}& 
\textbf{dn\_}& 
\textbf{yn\_}
\\
\hline
CPP \par macro& 
\textbf{kf}& 
\textbf{fs} \par & 
& 
&
& 
& 
 \\
\hline
\end{tabular}
\label{tab:tab1}
\end{center}
\end{table}
%--------------------------------------------------------------------------------------------------------------

N.B.   Parameter here, in not only parameter in the \textsc{Fortran} acceptation, it is also used for code variables 
that are read in namelist and should never been modified during a simulation. 
It is the case, for example, for the size of a domain (jpi,jpj,jpk).

\newpage
% ================================================================
% The program structure
% ================================================================
%\section{Program structure}
%abel{sec:Apdx_D_structure}

%To be done....
\end{document}
