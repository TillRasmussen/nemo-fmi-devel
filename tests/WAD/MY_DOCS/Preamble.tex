\documentclass[a4paper,11pt]{book}
%\documentclass[a4paper,11pt,makeidx]{book} <== may need this to generate index

%  makeindex NEMO_book  <== to regenerate the index
%  bibtex    NEMO_book	<== to generate  the bibliography

%\usepackage[french]{babel}
%\usepackage{color}
\usepackage{rotating, graphicx}			           % allows insertion of pictures
%\usepackage{graphics}			                   % allows insertion of pictures
\graphicspath{{Figures/}}                                  % Set global directory for pictures
\DeclareGraphicsExtensions{.pdf,.eps}                      % Use .eps for LaTeX2HTML
%\usepackage{xcolor}                                       % Incompatibility with color -> graphicx
\usepackage[capbesideposition={top,center}]{floatrow}      % allows captions
\floatsetup[table]{style=plaintop}                         % beside pictures
\usepackage[margin=10pt,font={small},                      % Gives small font for captions
labelsep=colon,labelfont={bf}]{caption}                    %
\usepackage{enumitem}                                      % allows non-bold description items
\usepackage{longtable}                                     % allows multipage tables
%\usepackage{colortbl}                                     % gives coloured panels behind table columns

%hyperref
\usepackage[pdftitle={NEMO ocean engine},pdfauthor={Gurvan Madec},pdfstartview=FitH,
bookmarks=true,bookmarksopen=true,breaklinks=true,colorlinks=true,
linkcolor=blue,anchorcolor=blue,citecolor=blue,filecolor=blue,menucolor=blue,urlcolor=blue]{hyperref}
% pdfsubject={The preprint document class
%   elsart},% pdfkeywords={diapycnal diffusion,numerical mixing,z-level models},%%  usage of exteranl hyperlink :  \href{mailto:my_address@wikibooks.org}{my\_address@wikibooks.org}
%                                                 \url{http://www.wikibooks.org}
%                                     or         \href{http://www.wikibooks.org}{wikibooks home}


%%%% page styles etc................
\usepackage{fancyhdr}
\pagestyle{fancy}
% with this we ensure that the chapter and section
% headings are in lowercase.
\renewcommand{\chaptermark}[1]{\markboth{#1}{}}
\renewcommand{\sectionmark}[1]{\markright{\thesection.\ #1}}
\fancyhf{}             % delete current setting for header and footer
\fancyhead[LE,RO]{\bfseries\thepage}
\fancyhead[LO]{\bfseries\hspace{-0em}\rightmark}
\fancyhead[RE]{\bfseries\leftmark}
\renewcommand{\headrulewidth}{0.5pt}
\renewcommand{\footrulewidth}{0pt  }
\addtolength{\headheight}{2.6pt}   % make space for the rule
%\addtolength{\headheight}{1.6pt}   % make space for the rule
% get rid of headers on plain pages
\fancypagestyle{plain}{\fancyhead{}\renewcommand{\headrulewidth}{0pt}}


%%%%  Section number in Margin.......
% typeset the number of each section in the left margin, with the start of each instance of
% sectional heading text aligned with the left hand edge of  the body text.
\makeatletter
\def\@seccntformat#1{\protect\makebox[0pt][r]{\csname the#1\endcsname\quad}}
\makeatother

% Leave blank pages completely empty, w/o header
\makeatletter
\def\cleardoublepage{\clearpage\if@twoside \ifodd\c@page\else
  \hbox{}
  \vspace*{\fill}
  \vspace{\fill}
  \thispagestyle{empty}
  \newpage
  \if@twocolumn\hbox{}\newpage\fi\fi\fi}
\makeatother

%%%% define the chapter  style ................
\usepackage{minitoc}				%In French : \usepackage[french]{minitoc}
%\usepackage{mtcoff}				% invalidate the use of minitocs
\usepackage{fancybox}

\makeatletter
\def\LigneVerticale{\vrule height 5cm depth 2cm\hspace{0.1cm}\relax}
\def\LignesVerticales{%
  \let\LV\LigneVerticale\LV\LV\LV\LV\LV\LV\LV\LV\LV\LV}
\def\GrosCarreAvecUnChiffre#1{%
  \rlap{\vrule height 0.8cm width 1cm depth 0.2cm}%
 \rlap{\hbox to 1cm{\hss\mbox{\color{white} #1}\hss}}%
  \vrule height 0pt width 1cm depth 0pt}
\def\GrosCarreAvecTroisChiffre#1{%
  \rlap{\vrule height 0.8cm width 1.6cm depth 0.2cm}%
 \rlap{\hbox to 1.5cm{\hss\mbox{\color{white} #1}\hss}}%
  \vrule height 0pt width 1cm depth 0pt}

\def\@makechapterhead#1{\hbox{%
   \huge
    \LignesVerticales
    \hspace{-0.5cm}%
    \GrosCarreAvecUnChiffre{\thechapter}
    \hspace{0.2cm}\hbox{#1}%
%    \GrosCarreAvecTroisChiffre{\thechapter}
%    \hspace{1cm}\hbox{#1}%
%}\par\vskip 2cm}
}\par\vskip 1cm}
\def\@makeschapterhead#1{\hbox{%
   \huge
    \LignesVerticales
    %\hspace{0.5cm}%
    \hbox{#1}%
}\par\vskip 2cm}
\makeatother

%\def\thechapter{\Roman{chapter}}   	% chapter number to be Roman


%%%%           Mathematics...............
%\documentclass{amsart}
\usepackage{xspace}                              % helpd ensure correct spacing after macros
\usepackage{latexsym,amssymb,amsmath}
\allowdisplaybreaks[1]				% allow page breaks in the middle of equations
\usepackage{TexFiles/Styles/math_abbrev}    % use maths shortcuts

\usepackage{times}			 	 % use times font for text
%\usepackage{mathtime}                          % font for illustrator to work (belleek fonts )
%\usepackage[latin1]{inputenc}                % allows some unicode removed (agn)


%%% essai commande
\newcommand{\nl}[1]{\texttt{\small{\textcolor{blue}{#1}}}}
\newcommand{\nlv}[1]{\texttt{\footnotesize#1}\xspace}
\newcommand{\smnlv}[1]{\texttt{\scriptsize#1}\xspace}

%%%% namelist & code display................................
\usepackage{alltt,verbatim}  	%%  alltt & verbatim for namelist
% namelists
\newcommand{\namdisplay}[1]{\begin{alltt}{\tiny\verbatiminput{Namelists/#1}}\end{alltt}\vspace{-10pt}}
\newcommand{\namtools}[1]{\begin{alltt}{\tiny\verbatiminput{Namelists/#1}}\end{alltt}\vspace{-10pt}}
% code display
%\newcommand{\codedisplay} [1] {\begin{alltt}{\tiny {\begin{verbatim} {#1}} \end{verbatim}} \end{alltt}      }



%%%% commands for working with text................................
% command to "comment out" portions of text ({} argument) or not ({#1} argument)
\newcommand{\amtcomment}[1]{}   	% command to "commented out" portions of text or not (#1 in argument)
\newcommand{\sgacomment}[1]{}   	% command to "commented out" portions of
\newcommand{\gmcomment}[1]{}   	% command to "commented out" portions of
%                                    				% text that span line breaks
%Red (NR) or Yellow(WARN)
%\newcommand{\NR}   {\colorbox{red}   {#1}}
%\newcommand{\WARN} {\colorbox{yellow}{#1}}



%%% index commands......................
\usepackage{makeidx}
%\usepackage{showidx}				% show the index entry

\newcommand{\mdl}[1]{\textit{#1.F90}\index{Modules!#1}}	%module (mdl)
\newcommand{\rou}[1]{\textit{#1}\index{Routines!#1}}	%module (routine)
\newcommand{\hf}[1]{\textit{#1.h90}\index{h90 file!#1}}	%module (h90 files)
\newcommand{\ngn}[1]{\textit{#1}\index{Namelist Group Name!#1}}	%namelist name (nampar)
\newcommand{\np}[1]{\textit{#1}\index{Namelist variables!#1}}        %namelist variable
\newcommand{\jp}[1]{\textit{#1}\index{Model parameters!#1}}	%model parameter (jp)
\newcommand{\pp}[1]{\textit{#1}\index{Model parameters!#1}} 	%namelist parameter (pp)
\newcommand{\ifile}[1]{\textit{#1.nc  }\index{Input NetCDF files!#1.nc}}	%input NetCDF files (.nc)
\newcommand{\key}[1]{\textbf{key\_#1}\index{CPP keys!key\_#1}}	%key_cpp (key)
\newcommand{\NEMO}{\textit{NEMO}\xspace}	%NEMO (nemo)

%%%%   Bibliography   .............
\usepackage[nottoc,notlof,notlot]{tocbibind}
\usepackage[square,comma]{natbib}
\bibpunct{[}{]}{,}{a}{}{;}                           %suppress "," after "et al."
\providecommand{\bibfont}{\small}

\usepackage{subfiles}                          % Separate compilation of chapters from whole manual
%\newcommand{\onlyinsubfile}[1]{#1}             % New commands for printing parts according to
%\newcommand{\notinsubfile}[1]{}                % the file being compiled
% Commands to use in the documentfile
%\renewcommand{\onlyinsubfile}[1]{}                      % Appears only if chapter   .tex file is compiled
%\renewcommand{\notinsubfile}[1]{#1}                     %    "    ""   "" NEMO_book .tex file is compiled

\DeclareMathAlphabet{\mathpzc}{OT1}{pzc}{m}{it}
